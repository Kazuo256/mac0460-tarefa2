\documentclass[a4paper,11pt]{article}
\usepackage[T1]{fontenc}
\usepackage[brazilian]{babel}
\usepackage[utf8]{inputenc}
\usepackage{lmodern}
\usepackage{multirow}

\title{MAC0460 - 2º semestre de 2012 \\ Relatório da Tarefa 2}
\author{Wilson Kazuo Mizutani (nº usp: 6797230)}

\begin{document}

\maketitle
%%\tableofcontents

\section{Dados Usados}

  O programa foi executado tendo como amostra de treinamento os dados no arquivo
  "sample" e como conjunto de testes os dados no arquivo "test", ambos entregues
  junto com esse trabalho.

\section{Teste de desempenho dos classificadores}

  Segue uma tabela resumindo o resultado do teste de eficiência dos
  classificadores. Para informações mais precisas, é só analisar a saída do
  programa "tarefa2.py".
  
  Cada classificador é identificado pelas distribuições que assume para cada
  classe, em ordem. Por exemplo, (U,U,N) é o classificador que assume que as
  distribuições das Classes 1, 2 e 3 são Uniforme, Uniforme e Normal,
  respectivamente. 
  
  Nas colunas das classes, o primeiro número é o número de vezes que o
  classificador daquela linha classificou alguém como sendo daquela classe, e o
  segundo número é o número de vezes que essa classificação foi de fato correta.
  Assim, por exemplo, se temos "25 (16)"\hspace{1pt} na coluna da "Classe 1"
  \hspace{1pt} de um certo classificador, significa que ele classificou 25
  elementos do conjunto de teste como sendo da classe 1, mas apenas 16 deles
  eram de fato da classe 1.
  
  Legenda:
    \begin{itemize}
      \item[\bf E:] { Distribuição exponencial }
      \item[\bf U:] { Distribuição uniforme }
      \item[\bf N:] { Distribuição normal }
    \end{itemize}
    
  \begin{tabular}{|c|c|c|c|c|}
    \hline
    Classificador  & Classe 1 & Classe 2 & Classe 3 & Total \\
    \hline
        E,E,E      &  25 (16) &  24 (20) &  11 (9)  &  45 \\
        E,E,U      &  11 (11) &  21 (17) &  28 (18) &  46 \\
        E,E,N      &  14 (13) &  14 (11) &  32 (17) &  41 \\
        E,U,E      &  25 (16) &  24 (20) &  11 (9)  &  45 \\
        E,U,U      &  11 (11) &  24 (20) &  25 (18) &  49 \\
        E,U,N      &  14 (13) &  24 (20) &  22 (17) &  50 \\
        E,N,E      &  25 (16) &  25 (20) &  10 (8)  &  44 \\
        E,N,U      &  11 (11) &  22 (18) &  27 (18) &  47 \\
        E,N,N      &  14 (13) &  24 (20) &  22 (17) &  50 \\
        U,E,E      &  31 (4)  &   0 (0)  &  29 (13) &  17 \\
        U,E,U      &  21 (2)  &  11 (0)  &  28 (18) &  20 \\
        U,E,N      &  19 (1)  &  13 (0)  &  28 (17) &  18 \\
        U,U,E      &   7 (2)  &  24 (20) &  29 (13) &  35 \\
        U,U,U      &  11 (11) &  24 (20) &  25 (18) &  49 \\
        U,U,N      &  11 (10) &  24 (20) &  25 (17) &  47 \\
        U,N,E      &   6 (2)  &  25 (20) &  29 (13) &  35 \\
        U,N,U      &   9 (9)  &  24 (18) &  27 (18) &  45 \\
        U,N,N      &  11 (10) &  24 (20) &  25 (17) &  47 \\
        N,E,E      &  43 (6)  &   0 (0)  &  17 (3)  &   9 \\
        N,E,U      &  21 (2)  &  11 (0)  &  28 (18) &  20 \\
        N,E,N      &  19 (1)  &  13 (0)  &  28 (17) &  18 \\
        N,U,E      &  19 (4)  &  24 (20) &  17 (3)  &  27 \\
        N,U,U      &  11 (11) &  24 (20) &  25 (18) &  49 \\
        N,U,N      &  13 (12) &  24 (20) &  23 (17) &  49 \\
        N,N,E      &  19 (4)  &  24 (20) &  17 (3)  &  27 \\
        N,N,U      &  11 (11) &  22 (18) &  27 (18) &  47 \\
        N,N,N      &  13 (12) &  24 (20) &  23 (17) &  49 \\
    \hline
  \end{tabular}
  
\section{Melhor classificador}

  Com base nos resultados observados, os classificadores com melhor desempenho
  foram (E,U,N) -- como era de se esperar -- e (E,N,N) -- que é relativamente
  parecido. Ambos foram os que mais acertaram ao classificar os dados do
  conjunto de testes (50 acertos). Seguem as matrizes de confusão desses dois
  classificadores:
  
  \vspace{12pt}
  
  \begin{tabular}{|cc|c|c|c|}
    \hline
    \multicolumn{2}{|c|}{Matriz de Confusão} &  
    \multicolumn{3}{|c|}{Respostas do Classificador} \\
    \cline{3-5}
    \multicolumn{2}{|c|}{do Classificador (E,U,N)}
    & Classe 1 & Classe 2 & Classe 3 \\
    \hline
    \multirow{3}{*}{Classes Reais}
    & \multicolumn{1}{|c|}{Classe 1} & 13 &  2 &  5 \\
    \cline{3-5}
    & \multicolumn{1}{|c|}{Classe 2} &  0 & 20 &  0 \\
    \cline{3-5}
    & \multicolumn{1}{|c|}{Classe 3} &  1 &  2 & 17 \\
    \hline
  \end{tabular}
  
  \begin{tabular}{|cc|c|c|c|}
    \hline
    \multicolumn{2}{|c|}{Matriz de Confusão} &  
    \multicolumn{3}{|c|}{Respostas do Classificador} \\
    \cline{3-5}
    \multicolumn{2}{|c|}{do Classificador (E,N,N)}
    & Classe 1 & Classe 2 & Classe 3 \\
    \hline
    \multirow{3}{*}{Classes Reais}
    & \multicolumn{1}{|c|}{Classe 1} & 13 &  2 &  5 \\
    \cline{3-5}
    & \multicolumn{1}{|c|}{Classe 2} &  0 & 20 &  0 \\
    \cline{3-5}
    & \multicolumn{1}{|c|}{Classe 3} &  1 &  2 & 17 \\
    \hline
  \end{tabular}
  
  \vspace{12pt}

  Sim, as duas matrizes de confusão ficaram iguais. Na verdade, ambos
  classificaram o conjunto de teste exatamente da mesma maneira, como pode ser
  constatado na saída do programa.

\end{document}
